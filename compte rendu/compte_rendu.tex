\documentclass[a4paper,10pt]{article}
\usepackage[utf8]{inputenc}
\usepackage[frenchkw]{algorithm2e}
\usepackage[francais]{babel}
\usepackage{graphicx}
\usepackage{amsfonts}
\usepackage{amsmath}

%opening
\title{}
\author{}

\begin{document}

\maketitle

\begin{abstract}
%a la fin
\end{abstract}

\section{Qu'est ce qu'un système de recommandation?}
\subsection{Notre approche}
Le but de notre projet est de recommander un film à un utilisateur spécifique. 
Pour cela, nous devons savoir à quel point un utilisateur aimera tel ou tel film pour lui recommander (ou pas). 
Il nous faudrait donc une sorte d'échelle d'affinité de l'utilisateur au film, ce qui parait pertinent est donc d'estimer une note qu'un utilisateur mettrait à un film s'il avait à le noter par exemple. 
Si cette note est au dessus d'un certain seuil, on lui recommandera (ou on recommandera le film ayant la meilleur note prédite par exemple). 
Notre approche du problème est donc de trouver un modèle mathématique permettant de trouver la note que mettrait un utilisateur à un film qu'il n'a pas vu. 
Cependant, sans aucunes connaissances des goûts l'utilisateur et des caractéristiques des films ceci est impossible. Il nous faut donc récupérer des données déjà existantes sur des utilisateurs et des films pour avoir une base et commencer à chercher une méthode.  
Nous avons donc récupérer des données.

\subsection{Première difficulté : les données}
\subsubsection{Extraction des données}
Le besoin d'avoir des données nous à guider vers une grande base de données appelé `` movielens '' qui est un site communautaire de recommandation de films où les utilisateurs du site notent des films de 0 à 5. 
Plusieurs jeux de données étaient disponibles et différé entre eux selon leur taille, 
Nous avons choisi de travailler avec une base de données de 670 utilisateurs (nu) et 9125 films (nf). 
On a donc extrait 2 fichiers : l’un contenant les 9125 films avec leurs titres et un numéro attribué , 
l’autre avec les notes des utilisateurs qui avaient chacun un identifiants, 
le fichier était du type : identifiant de l’utilisateur, id du film qu’il a noté, note.
Ces 2 fichiers étaient donc peu pratiques pour commencer à faire quelque chose avec, 
nous avons donc créé une fonction tableau\_des\_notes() qui permet de ranger toutes ces notes dans un tableau numpy nu*nf avec en lignes les utilisateurs, 
et en colonnes les films. Quand un utilisateur n’a pas vu un film et ne l'a donc pas noté, 
on insère un `` Nan ''(Not a Number) qui est un `` symbole '' (un peu vague) facile à traiter. On appellera ce tableau Y tout au long du projet.

\subsubsection{Analyse des données}
\subsection{}
\section{Formalisation du problème d'optimisation y compris modélisation}

Nous nous posons maintenant la question comment nous y prendre pour prédire des notes ?
Notre seul ressource est la matrice Y non pleine, sur quel modèle s'appuyer pour la compléter ?
C'est ce que nous allons étudier dans cette partie.

\subsection{Quelques notations}

Introduisons quelques notations qui seront utilisées par la suite dans ce rapport.\\

Nous travaillons tout au long de ce rapport avec des matrices réelles ce qui ne sera plus précisé. $\forall (n, p) \in \mathbb{N}^2$ nous notons l'ensemble des matrices réelles de dimension $n * p$ : ${\cal M}_{n, p}$\\
$\forall (n, p) \in \mathbb{N}^2$ $\forall A \in {\cal M}_{n, p}$ $\forall i \in \{1, 2, ..., n\}$ $a_i$ est la i-ème ligne de la matrice A et nous avons $a_i \in {\cal M}_{1, p}$. Et $\forall j \in \{1, 2, ..., p\}$ $a_{., j}$ est la j-ème colonne de la matrice A et nous avons $a_j \in {\cal M}_{n, 1}$. Remarquons que des majuscules sont utilisées pour des matrices qui ne sont ni des matrices lignes ni des matrices colonnes et des minuscules autrement. Nous utilisons aussi des minuscules pour les coefficients des matrices.\\
Sauf indication contraire, lorsque nous utilisons la lettre j il s'agit d'un entier naturel non nul et inférieur ou égal à $n_f$. De la même façon la lettre i représente, lorsque ça n'est pas précisé, un entier naturel non nul et inférieur ou égal à $n_u$.\\
Enfin, le film numéro j est le film dont les notes se trouvent à la j-ème colonne de Y et l'utilisateur i est l'utilisateur dont les notes qu'il a donné sont à la i-ème ligne de Y.

\subsection{Modélisation du problème : factorisation de matrices}

Posons $Y_f$ la matrice Y complétée qui contient toutes les prédictions, exactes, des notes qui nous manquent. C'est la matrice à laquelle nous voulons aboutir, que nous devons deviner. Voici notre méthode.\\ 
 
Nous supposons que nous sommes capable de déterminer à partir de Y deux matrices X et $\Theta$ telles que $\Theta X^T = Y_f$ avec $\Theta$ une matrice de dimension $n_u * n$ et X une matrice de dimension $n_f * n$. $n \in \mathbb{N}^*$ est quelconque, nous verrons par la suite que nous pouvons le fixer comme bon nous semble.\\ 
Notre hypothèse sur l'interprétation de $\Theta$ et X ? Ce qui nous fait supposer leur existence ? Nous allons éclairer ce point tout de suite par des explications plus précises.\\

Le paragraphe qui suit est la description de nos suppositions sur l'interprétation que nous pouvons faire à propos de n, $\Theta$ et X. Après chaque verbe au conditionnel il serait possible d'ajouter "d'après notre hypothèse".\\
Parlons un peu du nombre n. Ce dernier serait en fait un nombre de caractéristiques quelconques, qui nous sont inconnues et qui peuvent bien décrire nos films : si nous pensons qu'ils peuvent être décris efficacement par 10 caractéristiques, nous choisissons de trouver $\Theta$ et X avec $n = 10$. 
Une caractéristique peut être n'importe quoi, du taux d'action au taux de blondeur des cheveux de l'actrice principale. Pour l'instant admettons seulement que ces n caractéristiques peuvent exister. Nous les numérotons de 1 à n, puis nous verrons comment elles sont déterminées dans une autre partie.\\
Décrivons X. Nous avons déjà dit que cette matrice possède $n_f$ lignes. Chaque ligne caractériserait un film. Ainsi $\forall j \in \{1, 2, ..., n_f\}$ le film j serait décrit par $x_j$ de dimension $1 * n$. Comment ? En fait $\forall k \in \{1, 2, ..., n\}$ $x_{j,k}$ correspondrait à un taux de correspondance entre le film j et la k-ième caractéristique. Si la k-ième caractéristique est l'action et si le film j est un film d'action tandis que le film j' ($j' \in \{1, 2, ..., n_f\}$) est un film romantique sans action alors $x_{j,k}$ sera un réel supérieur à $x_{j',k}$.\\
Décrivons maintenant $\Theta$. $\forall i \in \{1, 2, ..., n_u\}$ $\theta_{i}$ représenterait les goûts de l'utilisateur i. $\forall k \in \{1, 2, ..., n\}$ $\theta_{i,k}$ serait un taux d'appréciation de la caractéristique k pour l'utilisateur i.\\
Ainsi seraient les deux matrices auxquelles nous voulons aboutir.\\

Voyons maintenant voir en quoi le produit $\Theta X^T$ peut valoir $Y_f$. D'après le produit matriciel, $\forall i \in \{1, 2, ..., n_u\}$ et $\forall j \in \{1, 2, ..., n_f\}$ on a $(y_{f})_{i,j} = theta_{i}(x_{j})^{T}$ : la note que l'utilisateur i a donné ou donnera au film j dépend des caractéristiques du film et du profil de l'utilisateur. Et nous avons :
\[(y_{f})_{i,j} = \sum_{k = 1}^{n} \theta_{i,k} * x_{j,k}\]
Ce qui semble logique : la note est une combinaison linéaire des taux de caractéristiques du film avec des coefficient qui sont plus ou moins élevé suivant le goût de l'utilisateur pour la caractéristique concernée.

\subsection{Fonction de coût}

Nous avons supposé que $\Theta$ et X existent, cependant il est très peu probable que ce soit le cas : nous pouvons arriver à deux matrices avec les dimensions voulues mais avec au mieux $\Theta X^T \approx Y_f$. Notre problème devient alors la minimisation de l'écart entre les notes de $\Theta X^T$ et celles de $Y_f$. Mais nous ne connaissons que quelques notes de $Y_f$ : les notes déjà données par les utilisateurs, celles de Y. Ce qui nous amène à introduire une fonction qui estime un écart entre les notes de $\Theta X^T$ et de Y :
\begin{align*}
J\colon{\cal M}_{n_u, n} \times {\cal M}_{n_f, n} &\longrightarrow \mathbb{R}^+\\
(\Theta, X)&\longmapsto \frac{1}{2}\sum_{\substack{i,j \\ y_{i,j} \ne nan}}(\theta_{i}(x_{j})^{T}-y_{i,j})^{2}
\end{align*}
Nous appelons J fonction de coût, c'est la fonction à minimiser pour trouver $\Theta$ et X optimaux qui représentent nos données le mieux possible.

\section{Implémentation algo}
%guillaume
\subsection{descente du gradient en general optimisation}
Nous allons proposer une méthodologie pour minimiser 
\begin{algorithm}[H]
 \Donnees{this text}
 \Res{how to write algorithm with \LaTeX2e }
 initialization\;
 \Repeter{not at end of this document}{
  read current\;
  \eSi{understand}{
   go to next section\;
   current section becomes this one\;
   }{
   go back to the beginning of current section\;
  }
 }
 \caption{How to write algorithms}
\end{algorithm}
\subsubsection{ecrire algo puis dire comment on a fait au début}
\section{Mise en application}
\subsection{Notre approche pour familiariser au probleme}
%juliette
\subsection{yassine}
\subsubsection{taux d'erreur}
\subsubsection{influence des parametres}
\subsection{limites de notre approche}
%yassine
\subsubsection{Est-ce qu'on arrive a reccomander un film a un utilisateur}
%guillaume
\section{conclusion}
%1 page

\subsection{perspectives}
%guillame
\end{document}