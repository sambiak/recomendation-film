\documentclass[a4paper,10pt]{article}
\usepackage[utf8]{inputenc}
\usepackage[francais]{babel}
\usepackage[frenchkw]{algorithm2e}
\usepackage[T1]{fontenc}

%opening
\title{}
\author{}

\begin{document}

\maketitle

\begin{abstract}
%a la fin
\end{abstract}

\section{Qu'est ce qu'un système de reccomandation?}
\subsection{Première difficulté : les données}
%juliette bien poser y
\subsection{Notre approche}
Le but de notre projet est de recommander un film à un utilisateur spécifique. 
Pour cela, nous devons savoir à quel point un utilisateur aime tel ou tel film pour lui recommander (ou pas). 
Il nous faudrait donc une sorte d'échelle d'affinité de l'utilisateur au film, ce qui parait pertinent est donc d'estimer une note qu'un utilisateur mettrait à un film s'il avait à le noter par exemple. 
Si cette note est au dessus d'un certain seuil, on lui recommandera (ou on recommandera le film ayant la meilleur note prédite par exemple). 
Notre approche du problème est donc de trouver un modèle mathématique permettant de trouver la note que mettrait un utilisateur à un film qu'il n'a pas vu. 
Cependant, sans aucunes connaissances des gouts l'utilisateur et des caractéristiques des films ceci est impossible. Il nous faut donc récuperer des données déjà existentes sur des utilisatuers et des films pour avoir une base et commencer à chercher une méthode.  
Nous avons donc récuperer des données.
\subsection{}
\section{Formalisation du problème d'optimisation y compris modelisation}
%antho
\subsection{Factorisation}
%Formalisation
\subsection{Fonction cout}
%toujours antho
\subsubsection{Pourquoi on cherche a minimiser}
\section{Implémentation algo}
%guillaume
\subsection{descente du gradient en general optimisation}
Nous allons proposer une méthodologie pour minimiser 
\begin{algorithm}[H]
 \Donnees{this text}
 \Res{how to write algorithm with \LaTeX2e }
 initialization\;
 \Repeter{not at end of this document}{
  read current\;
  \eSi{understand}{
   go to next section\;
   current section becomes this one\;
   }{
   go back to the beginning of current section\;
  }
 }
 \caption{How to write algorithms}
\end{algorithm}
\subsubsection{ecrire algo puis dire comment on a fait au début}
\section{Mise en application}
\subsection{Notre approche pour familiariser au probleme}
%juliette
\subsection{yassine}
\subsubsection{taux d'erreur}
\subsubsection{influence des parametres}
\subsection{limites de notre approche}
%yassine
\subsubsection{Est-ce qu'on arrive a reccomander un film a un utilisateur}
%guillaume
\section{conclusion}
%1 page

\subsection{perspectives}
%guillame
\end{document}